\chapter{Methodology}\label{chap:methodology}
\section{Overview}
This chapter describes all processes of this experiment. There are six steps, and each step has a corresponding goal. The steps are shown in Fig\ref{fig:method}. 


\section{Experiment}\label{sec:experiment}

\section{Literature Review and Research}
In this step, we reviewed the open literature, including block ciphers, KATAN, several attacks and BTC. We went through this step to explore whether BTC can make the results of Boomerang attack more credible, and determine how to improve BTC.

\section{Define the SMT model}
In this step, we analyze KATAN's cryptographic primitives, write Python code that can run on CryptoSMT, and then run differential analysis tests on CryptoSMT to ensure that the code is runnable. This step, we obtain a usable SMT model of KATAN.

\section{Optimize the smt solver with BTC tools}\label{m3}
In this step, we code the smt solver with BTC, and use data that be generated randomly to test and investigate the smt solver over multiple rounds, in the end, we get a boomerang switch.

\section{Find the best differential trails}\label{m4}
In this step, we use the boomerang switch that be created in \ref{m3}, to find the best $E_0$ and $E_1$ and a multiple round $E_m$. In the end, we can crate a verified boomerang distinguisher for KATAN.

\section{Cover the key}
In this step, we generate plaintext and a key randomly, and use the data to cover the key with the boomerang distinguisher that be created in \ref{m4}. In this end, we design a scheme for recover key and recover the key successfully.

\section{Proposed A tools}
Last, when we create a tool using Python that includes the distinguisher we proposed in \ref{m4}, it can automatically generate differential paths for KATAN.


\section{Chapter Summary}
This Chapter describes our experimental steps and our desired goals.